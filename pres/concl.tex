\begin{frame}
\frametitle{Conclusions}
	\begin{textblock*}{12cm}(0.4cm,2cm) % {block width} (coords)
	\begin{block}{Main outcomes}
		\begin{enumerate}
			\itemsep=3mm
			\item The neutron energy spectrum at the beginning-of-life 
			(BOL) for the \gls{TAP} eactor is \textbf{fast} $\Rightarrow$ 
			\textbf{gas removal system can be disabled} at \gls{BOL}
			\item The spectrum becomes \textbf{more thermal} during operation 
			due to \textbf{increasing moderator-to-fuel ratio} $\Rightarrow$ 
			the xenon gas removal system \textbf{must operate} to enable 
			load-following

			\item Multiplication factor during depletion simulations for 
			postulated load-following transient demonstrated following 
			dynamics:
				\begin{itemize}
					\item For PWR, dropped rapidly after shutdown; reached 
					maximum poisoning effect ($-1500pcm$) $\approx$ 7 hours 
					after shutdown
					\item For \gls{TAP} concept, very small change in 
					$k_{eff}$, no effect of $^{135}$Xe poisoning was observed
				\end{itemize}
			\item \textbf{PWR}: the drop happened because 			
			$m_{^{135}I}/m_{{135}Xe} = 2.3$ and $^{135}$I decays to $^{135}$Xe 
			faster ($\tau_{1/2}=6.6h$) than $^{135}$Xe decays to $^{135}$Cs 
			($\tau_{1/2}=9.17h$)
			\item \textbf{\gls{TAP} \gls{MSR}}: no poisoning effect because 
			$m_{^{135}I}/m_{{135}Xe} = 0.9$
		\end{enumerate}
	\end{block}


	\end{textblock*}
\end{frame}

\begin{frame}
\frametitle{Future work}
\begin{textblock*}{12cm}(0.4cm,2.0cm) % {block width} (coords)
	\begin{block}{Future research effort}
		\begin{enumerate}
		\itemsep=5mm
			\item Investigate the impact of xenon poisoning for the TAP 
			concept at the end-of-life (EOL), which night have softer neutron 
			spectrum
			\item Take into account gas removal system using the online 
			reprocessing tool SaltProc \cite{rykhlevskii_arfc/saltproc_2018, 
			rykhlevskii_modeling_2019}
			\item Take into consideration the \gls{TAP} design adjustable 
			moderator-to-fuel ratio
			\item Develop a fuel processing system that enables load-following 
			in a various commercial thermal molten salt reactors:
				\begin{itemize}
					\item Terrestrial Energy Integral Small Modular Reactor
					\item ThorCon Small Modular Reactor
				\end{itemize}
			\item Analyze multi-physics transients using the coupled 
			neutronics/thermal-hydraulics code Moltres 
			\cite{lindsay_introduction_2018}
		\end{enumerate}
	\end{block}
\end{textblock*}
\end{frame}

\begin{frame}
\frametitle{Acknowledgement}
\begin{itemize}
	\item Andrei Rykhlevskii, Kathryn Huff, and Tomasz 	Kozlowski are 
	supported by DOE ARPA-E MEITNER program award DE-AR0000983.
	\item This research is part of the Blue Waters sustained-petascale 
	computing project, which is supported by the National Science Foundation 
	(awards OCI-0725070 and ACI-1238993) and the state of Illinois.
	\item Kathryn Huff is additionally supported by the NRC Faculty 
	Development Program, the NNSA (awards DE-NA0002576 and DE-NA0002534), and 
	the International Institute for Carbon Neutral Energy Research 
	(WPI-I2CNER).
	\item The authors would like to thank  members of Advanced Reactors and 
	Fuel Cycles	research group (ARFC) at the University of Illinois at Urbana 
	Champaign who provided valuable reviews and proofreading.
	\item Anshuman Chaube, Alvin Lee (University of Illinois at 
	Urbana-Champaign).
\end{itemize}
\begin{figure}[t]
	\hspace*{-0.4in}
	\includegraphics[height=0.25\textheight]{./images/acks.png}
\end{figure}
\end{frame}